\chapter{Documentation}

\section{Gym Environments}

WIP

\section{AbstractPortfolioEnvWithTCost}

WIP

\begin{minted}{python3}
  class AbstractPortfolioEnvWithTCost(gym.Env):
  def __init__(self, w_lb=0.0, w_ub=1.0, cp=0.0, cs=0.0, logging=True):
    # set constants
    self.eta = 1 / 252
    self.cp, self.cs = cp, cs
    self.logging = logging

    # get data, set problem size
    self.num_time_periods, self.universe_size = self.get_data()

    # set spaces
    assert w_lb <= w_ub
    self.observation_space = self.get_obs_space()
    self.action_space = gym.spaces.Box(
      low=w_lb,
      high=w_ub,
      shape=(self.universe_size + 1,),
      dtype=np.float32
    )

  @abstractmethod
  def get_obs_space(self) -> gym.spaces.Box:
    """Result is assigned to ``self.observation_space``"""
    pass

  @abstractmethod
  def get_data(self) -> tuple[int, int]:
    """
    This abstract function loads/fetches state data and stores it on the environment.
    This will be called during initialization. The properties assigned here should be
    accessed into the __compute_state() method. Note that the data should provide for
    one more than the number of time periods desired (for the initial state).
    :return: (number of time periods, number of stock tickers)
    """
    pass

  @abstractmethod
  def get_state(self) -> npt.NDArray[float]:
    """
    Computes and returns the new state at time ``self.t`` (to be used for
    calculating weights at the start of time period ``self.t+1``).
    When ``self.t == 0``, it should output the initial state.
    """
    pass

  @abstractmethod
  def get_prices(self) -> npt.NDArray[float]:
    """
    Obtains the security prices at time ``self.t`` (at the
    beginning of time period ``self.t+1``). When ``self.t == 0``, it
    should output the initial prices.
    """
    pass

  def initialize_reward(self):
    self.A, self.B = 0.0, 0.0

  def compute_reward(self) -> float:
    R = np.log(self.new_port_val / self.port_val)
    dA = R - self.A
    dB = R ** 2 - self.B
    if self.B - self.A ** 2 == 0:
        D = 0
    else:
        D = (self.B * dA - 0.5 * self.A * dB) / (self.B - self.A ** 2) ** (3 / 2)
    self.A += self.eta * dA
    self.B += self.eta * dB
    return D

  def find_mu(self, w_old: npt.NDArray[float], w_new = npt.NDArray[float]) -> float:
    cp, cs = self.cp, self.cs

    def f(mu: float) -> float:
        return ((1 - cp * w_new[-1] - (cs + cs - cs*cp) * (w_new[:-1] - mu * w_old[:-1]).clip(min=0).sum()) /
                (1 - cp * w_old[-1]))

    mu = 0.0
    for _ in range(30):
        mu = f(mu)
    return mu

  def step(self, action: npt.NDArray[float]) -> tuple:
    action = action / action.sum()
    self.w_new = action
    self.t += 1
    self.v_new = self.get_prices()
    self.y = self.v_new / self.v
    self.mu = self.find_mu(self.y * self.w / (self.y * self.w).sum(), self.w_new)
    self.new_port_val = self.port_val * self.mu * (self.y @ self.w)

    self.reward = self.compute_reward()

    self.state = self.get_state()
    self.w = self.w_new
    self.v = self.v_new
    self.port_val = self.new_port_val

    if self.logging:
      info = {
        'port_val': self.port_val
      }
    else:
      info = {}

    finished = (self.t == self.num_time_periods)
    return self.state.copy(), self.reward, finished, False, info

  def reset(self, *args, **kwargs) -> tuple[np.ndarray, dict]:
    # portfolio weights (final is cash weight)
    self.w = np.zeros(self.universe_size + 1, dtype=float)
    self.w[-1] = 1.0

    self.port_val = 1.0

    self.initialize_reward()

    # compute and return initial state
    self.t = 0
    self.state = self.get_state()
    self.v = self.get_prices()
    return self.state.copy(), {}
\end{minted}

