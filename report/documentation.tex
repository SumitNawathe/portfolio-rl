\chapter{Documentation}

\section{Gym Environments}

Gymnasium is an open-source reinforcement learning environment.
Its goal is to standardize the interface between reinforcement learning algorithms
and their environments.
Much of the information for this section is taken from their documentation: https://gymnasium.farama.org/

The extendable base class for the gym environment is \texttt{gym.Env}. A subclass must define the following two methods:
\begin{itemize}
  \item \texttt{reset()} initializes the environment. Its return type is \texttt{tuple[ObsType, dist[str, Any]]}.
  The first element is the initial state, and the dictionary is for intitial logging information.
  \item \texttt{step(a: ActType)} performs one update of the environment where the agent takes action \texttt{a}.
  Its return type is \texttt{tuple[ObsType, float, bool, bool, dist[str, Any]]}. In order, the elements are:
  (1) the next state provided to the agent, (2) the reward value for that step, (3) whether the environment has terminated,
  (4) whether the environment has truncated, and (5) a dictionary of logging information.
\end{itemize}

\texttt{ObsType} and \texttt{ActType} are the types of the observation (state) space and action spaces, respectively,
which must set within the environment initialization. There are a few standard types of spaces supported by Gymnasium:

\begin{itemize}
  \item \texttt{gym.spaces.Discrete(n)}: A set of \texttt{n} distinct discrete values.
  \item \texttt{gym.spaces.Box(low, high, shape, dtype)}: A generalized rectangle. The dimensions are
  specified by the \texttt{shape} parameter, and all entries have \texttt{dtype} type (generally \texttt{np.float64}). Each component
  can have its own upper and lower bound if they are tuples (if a single float is provided, that bound is used for all components).
  \item \texttt{gym.spaces.Tuple(spaces)}: A space that is a tuple (cartesian product) of other spaces.
\end{itemize}

It is possible to define custom spaces, but standard RL training algorithms and models
generally only support select compositions of the standard spaces.

\section{RL Framework}

In order to ensure consistency and modularity, we have created a framework of classes
that allows us to easily swap components of our RL environment to test various data
and reward strategies.

\subsection{AbstractRewardManager}

This abstract class manages the calculation of rewards for the RL agent.
Its interface is minimal. Subclasses can optionally have a constructor.

\begin{minted}[escapeinside=||,mathescape=true,linenos,xleftmargin=20pt]{python3}
from abc import ABC, abstractmethod

class AbstractRewardManager(ABC):
  @abstractmethod
  def initialize_reward(self):
    """
    Initializes all constants used in the reward calculation.
    Must be called at least once before compute_reward() is called.
    Should be called in the environment's reset() method.
    """
    pass

  @abstractmethod
  def compute_reward(self, old_pval: float, new_pval: float) -> float:
    """
    Computes the reward based on the old and new portfolio values.
    Should be called in the environment's step() method.
    """
    pass
\end{minted}


\subsection{AbstractDataManager}

This abstract class manages reading and iterating through the state and price data for the environment.

\begin{minted}[escapeinside=||,mathescape=true,linenos,xleftmargin=20pt]{python3}
from abc import ABC, abstractmethod
import numpy as np
import numpy.typing as npt
import gymnasium as gym

class AbstractDataManager(ABC):
  @abstractmethod
  def get_obs_space(self) -> gym.spaces.Box:
    """Result is assigned to the environment's observation space"""
    pass

  @abstractmethod
  def get_data(self) -> tuple[int, int]:
    """
    This function loads/fetches state data from files and stores it.
    Should be called in the environment's reset() method.
    The properties assigned here should be accessed in get_state().
    Note that the data should provide for one more than the number
      of time periods desired (for the initial state).
    Returns: (number of time periods, number of stock tickers)
    """
    pass

  @abstractmethod
  def get_state(
    self,
    t: int,
    w: npt.NDArray[np.float64],
    port_val: np.float64
  ) -> npt.NDArray[np.float64]:
    """
    Computes and returns the new state at time t.
    State can include the current portfolio weight and value,
      provided as additional parameters.
    This state will be used by the agent for calculating weights
      at the start time time period t+1.
    When t=0, it should output the initial state.
    """
    pass

  @abstractmethod
  def get_prices(self, t: int) -> npt.NDArray[np.float64]:
    """
    Returns the security prices at time t (at the beginning
      of time period t+1).
    When t=0, it should output the initial prices.
    """
    pass
  \end{minted}

\subsection{PortfolioEnvWithTCost}

This class is the main gym environment.
It iterates through the data and rewards using the \texttt{AbstractDataManager} and \texttt{AbstractRewardManager} provided as arguments.
To compute the portfolio updates, it employs the transaction costs-cognizant approach described in section \ref{transaction_costs_section}.

Let \texttt{w} be the portfolio weights. We use the convention that \texttt{w[-1]} is the weight for the risk-free asset,
and \texttt{w[:-1]} are the weights for the risky assets. (Note that this is a different convention than the one used in \cite{drl_framework}.)

We omit much of the actual code and instead provide pseudocode or equations where relevant.

\begin{minted}[escapeinside=||,mathescape=true,linenos,xleftmargin=20pt]{python3}
from abc import ABC, abstractmethod
import numpy as np
import numpy.typing as npt
import gymnasium as gym

class PortfolioEnvWithTCost(gym.Env):
  def __init__(
    self,
    dm: AbstractDataManager,
    rm: AbstractRewardManager,
    w_lb=0.0, w_ub=1.0,
    cp=0.0, cs=0.0,
    logging=True
  ):
    """
    Initializes the environment with the given managers and constants.
    w_lb and w_ub are the lower and upper bounds for the portfolio weights.
    cp and cs are commision weights for purchasing and selling assets.
    If logging is enabled, step() will return the portfolio value.
    """
    # register managers
    # set constants from given arguments

    # get data from manager
    self.num_time_periods, self.universe_size = self.dm.get_data()

    # set environment observation and action spaces
    assert w_lb <= w_ub
    self.observation_space = self.dm.get_obs_space()
    self.action_space = gym.spaces.Box(
      low=w_lb,
      high=w_ub,
      shape=(self.universe_size + 1,),
      dtype=np.float64
    )

  def find_mu(
    self,
    w_old: npt.NDArray[np.float64],
    w_new = npt.NDArray[np.float64]
  ) -> float:
    """
    Uses the iterative technique to find mu for the given old weights
    (after returns), new weights, and the transaction cost commision rates.
    """
    pass

def step(self, action: npt.NDArray[np.float64]) -> tuple:
  """
  Performs one time step update using the agent's action.
  Returns: a tuple of
    - copy of the new state
    - reward for the action
    - whether the environment has terminated (reached end of data)
    - whether the environment has truncated (always False)
    - information dictionary (possibly containing portfolio value)
  """
  # check that the action is normalized
  # find the value of mu using the helper function
  # perform the weight and portfolio value updates
  # obtain the reward uding the reward manager
  # obtain the new state using the data manager
  # update all instance variables
  # if logging enabled, return the portfolio value in addition to
  pass

def reset(self, *args, **kwargs) -> tuple[np.ndarray, dict]:
  """
  Resets the environment to an initial state.
  Returns: the initial state and an empty dictionary (required by gym).
  """
  # reset portfolio weights
  # initialize reward manager
  # obtain initial state and prices from data manager
  pass
\end{minted}

\subsection{BasicPortfolioWithTCost}

\cite{drl_mvo} provides a simple yet effective state space setup augmented with 
volatility indicators that characterize market trends. We provide a brief description here.

Let asset $i$'s price at time $t$ be denoted $P_{i, t}$. Then the one-period simple return of the asset 
is $R_{i, t} = \frac{P_{i, t} - P_{i, t-1}}{P_{i, t-1}}$ and the one-period gross return is $\frac{P_{i, t}}{P_{i, t-1}} = R_{i, t} + 1$. 
Thus, one-period log returns can be denoted $r_{i, t} = \log(R_{i, t} + 1)$. Given a universe of 
$n$ assets and cash denoted by $c$, the authors form the subsequent $[(n + 1) \times T]$ state matrix:
\begin{align*}
  S_t = \begin{bmatrix}
    w_1 & r_{1, t-1} & \cdots & r_{1, t - T + 1}\\
    w_2 & r_{2, t-1} & \cdots & r_{2, t - T + 1}\\
    \vdots & \vdots & \ddots & \vdots\\
    w_n & r_{n, t-1} & \hdots & r_{n, t - T + 1}\\
    w_c & vol_{20} & \frac{vol_{20}}{vol_{60}} & VIX_t \ldots
  \end{bmatrix}
\end{align*}
where the first column is the portfolio weights of each asset at the beginning of time $t$, which 
can differ slightly from the portfolio weights at the end of time $t-1$. The last row features three
market volatility indicators --- $vol_{20}, \frac{vol_{20}}{vol_{60}}$, and $VIX_t$ --- which are 
particularly important given the authors' imposed long-only and no-leverage constraints.
\begin{align*}
  w_c \geq 0, \ w_i \geq 0 \:|\: \forall \ i&&
  w_c + \sum_{i=1}^n w_i = 1
\end{align*}
The first indicator, $vol_{20}$, is the $20$-day rolling window standard deviation of daily S$\&$P500 returns. 
Similarly, $vol_{20}$ is the $60$-day rolling window standard deviation of daily S$\&$P500 returns. Thus, their ratio 
is given by $\frac{vol_{20}}{vol_{60}}$. If $\frac{vol_{20}}{vol_{60}} < 1$, then the
last $20$ days have been less volatile than the last $60$ days, which suggests a shift from a high volatility 
market trend to a lower volatility market trend. Finally, the last indicator $VIX$ is a measure 
of the stock market's expected volatility.

\begin{minted}[escapeinside=||,mathescape=true,linenos,xleftmargin=20pt]{python3}
  import numpy as np
  import pandas as pd
  import gymnasium as gym
  from portfolio_env_with_tcost import PortfolioEnvWithTCost
  import numpy.typing as npt
  import gymnasium as gym

  class PortfolioEnvWithTCost(PortfolioEnvWithTCost):
    def get_obs_space(self) -> gym.spaces.Box:
        return gym.spaces.Box(low=-np.inf, high=np.inf, shape=(self.universe_size+1, 
        100+1), dtype=np.float32)

    def get_data(self) -> tuple[int, int]:
        # read SNP data
        df = pd.read_csv('crsp_snp100_2010_to_2024.csv', dtype='string')
    
        # convert datatypes
        df = df[['date', 'TICKER', 'PRC', 'VOL', 'ASKHI', 'BIDLO', 'FACPR']]
        df.date = pd.to_datetime(df.date)
        df.FACPR = df.FACPR.fillna('0.0')
        df.astype({
            'PRC': float,
            'VOL': float,
            'ASKHI': float,
            'BIDLO': float,
            'FACPR': float
        })
    
        # drop duplicates and nans
        df = df.drop_duplicates(subset=['date', 'TICKER'])
        df.dropna(inplace=True)
    
        # only include stocks that are present in all dates
        ticker_ok = df.TICKER.value_counts() == df.TICKER.value_counts().max()
        def is_max_val_count(ticker: str) -> bool:
          return ticker_ok[ticker]
        ok = df.apply(lambda row: is_max_val_count(row['TICKER']), axis=1)
        df = df[ok]
        df = df[(df.date.dt.year >= 2010) & (df.date.dt.year <= 2019)]
    
        # create stock array
        self.stock_df = df.pivot(index='date', columns='TICKER', values='PRC').astype(float)
        
        # adjust for stock splits
        facpr_df = df.pivot(index='date', columns='TICKER', values='FACPR').astype(float)
        self.stock_df = self.stock_df * (1+facpr_df).cumprod(axis=0)
        # assert np.all(self.stock_df.pct_change().iloc[1:, :] > -1), 
        f"{(self.stock_df.pct_change().iloc[1:, :] <= -1).sum().sum()=}, 
        {np.any(pd.isna(self.stock_df.pct_change().iloc[1:, :]))}"
        self.ret = np.log(self.stock_df.pct_change().iloc[1:, :] + 1)
    
        # get times and dickers
        self.times = df.date.unique()[1:]
        self.tickers = df.TICKER.unique()
    
        # read index data and compute volatilities
        idx_df = pd.read_csv('crsp_snpidx_2010_to_2024.csv', dtype={
          'DATE': 'string',
          'vwretd': float
        })
        idx_df.DATE = pd.to_datetime(idx_df.DATE)
        idx_df['vol_20'] = idx_df.vwretd.rolling(20).std()
        idx_df['vol_60'] = idx_df.vwretd.rolling(60).std()
        idx_df.set_index('DATE', inplace=True)
        self.vol_20 = idx_df.vol_20
        self.vol_60 = idx_df.vol_60
    
        # get vix data
        vix_df = pd.read_csv('crsp_vix_2010_to_2024.csv', dtype={
          'Date': 'string',
          'vix': float
        })
        vix_df.Date = pd.to_datetime(vix_df.Date)
        vix_df.set_index('Date', inplace=True)
        self.vix_df = vix_df.vix
        
        return len(self.times)-100-1, len(self.tickers)
    
    def get_state(self) -> npt.NDArray[np.float64]:
        # today is self.times[self.t+100]
        s = np.zeros((self.universe_size+1, 100+1))
        s[:, 0] = self.w
        # s[1:, :-1] = self.ret[self.t:self.t+100, :].T
        # 100 past returns, up to yesterday
        s[1:, :-1] = self.ret.loc[self.times[self.t:self.t+100], :].to_numpy().T
        s[-1, 1] = self.vol_20[self.times[self.t+100-1]] # yesterday's vol_20
        s[-1, 2] = self.vol_20[self.times[self.t+100-1]] / 
                    self.vol_60[self.times[self.t+100-1]] # yesterday's vol ratio
        s[-1, 3] = self.vix_df[self.times[self.t+100-1]] # yesterday's vix
        return s

    def get_prices(self) -> npt.NDArray[np.float64]:
        # today is self.times[self.t+100]
        return np.append(self.stock_df.loc[self.times[self.t+100], :].to_numpy()
        .flatten(), 1.0)
  
\end{minted}

\subsection{MPTWithTCost}

Covariance and correlation matrices of historical stocks returns are the core of modern portfolio theory; 
indeed the investor 
who incorporates the covariance matrix can effectively minimize portfolio variance. 
\cite{drl_modern_portfolio_theory} integrates correlation matrices and price-based technical 
indicators into an RL-setting, effectly combining RL with modern portfolio theory. We provide a brief overview 
of their methodology and infrastructure.

Technical indicators are pattern-based signals used to describe the current state 
of an asset's price based on previous trends. The authors use asset price and three technical indicators to construct 
tensors: moving average (MA), relative strength index (RSI), and moving avarage convergence divergence (MACD).

MA computes the average price over a rolling window. Let $P_{i,t}$ be the close 
price for asset $i$ at time $t$ and let $w = 28$ be the window. 
\[MA_w(t) = \frac{\sum_{t=0}^{w-1}P_{i, t}}{w}\] 
RSI describes the upward or downward pressure on an asset's close price.
Let $w=14$ and consider an exponential moving average 
\[EMA_w(t) = \alpha P(t) + (1-\alpha) * EMA_w(t-1), \alpha = \frac{2}{1 + w}\]
Then RSI is computed as 
\[RSI_t(w) = 100 * \Bigl(1 - \frac{1}{1 + \frac{EMA_w(\max(P_t - P_{t-1}, 0))}{EMA_w(\min(P_t - P_{t-1}, 0))}}\Bigr)\]
Finally, MACD describes the relationship between two EMAs. Let $k = 26, d=12, w=9$
\[MACD_w(t) = \sum_{i=1}^n EMA_k(i) - \sum_{i=1}^w EMA_d(i)\]

Using these price and technical indicator matrices, construct tensors 
\begin{align*}
  V_t = [V_{\text{t, close}}, V_{\text{t, MA}}, V_{\text{t, RSI}}, V_{\text{t, MACD}}] \in \mathbb{R}^{4 \times m \times n}\\
  Cor_t = [Cor_{\text{t, close}}, Cor_{\text{t, MA}}, Cor_{\text{t, RSI}}, Cor_{\text{t, MACD}}] \in \mathbb{R}^{4 \times n \times n}
\end{align*}
For indicator $i$ and asset $n$, the authors use the following trasnformation to fuse the properties 
into one tensor. 
\[F_{t(i, n)} = V_{t(i, n)} Cor^T_{t(i, n)}, F_{t(i, n)} \in R^{m \times n}\]
Now for all indicators and assets, we can efficiently compute the transformed tensor to be $F_t \in R^{4 \times n \times m \times n}$.
This transformed tensor then goes through a convolution layer with 32 feature maps and a $(1, 3, 1)$ kernel and then ReLU activation try 
try and capture spacial features.
\[F_t' = \text{Conv3D}(F_t), F_t' \in \mathbb{R}^{32 \times n \times m \times n}\]
where the equal shape of $F_t$ and $F_t'$ can be achieved using "equal" padding. Finally, the Tucker Decomposition is used
to decompose $F_t'$ into a core tensor $C$ and a list of factor matrices $M$. Tucker Decomposition is also recognized as a higher order Singular Value Decomposition. 
\[C, M = \text{Tucker Decomposition}(F_t'), C \in \mathbb{R}^{1 \times 1 \times n \times m} \implies C \in \mathbb{R}^{n \times m}\]
Setting $C = \mathcal{S}$ the new state space at time, the agent proceeds with the learing process
using the DDPG algorithm, which we have aforementioned.

\begin{minted}[escapeinside=||,mathescape=true,linenos,xleftmargin=20pt]{python3}

  import numpy as np
  import pandas as pd
  import gymnasium as gym
  import pandas_ta as ta
  from tensorly.decomposition import Tucker
  tensorly.set_backend('pytorch')
  import torch
  from portfolio_env_with_tcost import PortfolioEnvWithTCost
  from typing import Tuple, Optional
  import numpy.typing as npt

  class MPTWithTCost(PortfolioEnvWithTCost):

    def get_obs_space(self) -> gym.spaces.Box:
        self.state_shape = (1, 1, 28, self.universe_size)
        self.t = 0
        self.get_indicators()
        return gym.spaces.Box(low=-np.inf, high=np.inf, shape=self.state_shape, 
        dtype=np.float64)
    
    def get_data(self) -> Tuple[int, int]:
        # read SNP data
        df = pd.read_csv('crsp_snp100_2010_to_2024.csv', dtype='string')
    
        # convert datatypes
        df = df[['date', 'TICKER', 'PRC', 'VOL', 'ASKHI', 'BIDLO', 'FACPR']]
        df.date = pd.to_datetime(df.date)
        df.FACPR = df.FACPR.fillna('0.0')
        df.astype({
            'PRC': float,
            'VOL': float,
            'ASKHI': float,
            'BIDLO': float,
            'FACPR': float
        })
    
        # drop duplicates and nans
        df = df.drop_duplicates(subset=['date', 'TICKER'])
        df.dropna(inplace=True)
    
        # only include stocks that are present in all dates
        ticker_ok = df.TICKER.value_counts() == df.TICKER.value_counts().max()

        def is_max_val_count(ticker: str) -> bool:
          return ticker_ok[ticker]
        
        ok = df.apply(lambda row: is_max_val_count(row['TICKER']), axis=1)
        df = df[ok]
        df = df[(df.date.dt.year >= 2010) & (df.date.dt.year <= 2019)]
    
        # create stock array
        self.stock_df = df.pivot(index='date', columns='TICKER', values='PRC').astype(float)
        
        idx_df = pd.read_csv('crsp_snpidx_2010_to_2024.csv', dtype={
          'DATE': 'string',
          'vwretd': float
        })
        idx_df.DATE = pd.to_datetime(idx_df.DATE)
        idx_df['vol_20'] = idx_df.vwretd.rolling(20).std()
        idx_df['vol_60'] = idx_df.vwretd.rolling(60).std()
        idx_df.set_index('DATE', inplace=True)
        self.idx_df = idx_df

        # adjust for stock splits
        facpr_df = df.pivot(index='date', columns='TICKER', values='FACPR').astype(float)
        self.stock_df = self.stock_df * (1+facpr_df).cumprod(axis=0)
        self.ret = np.log(self.stock_df.pct_change().iloc[1:, :] + 1)
        self.times = df.date.unique()[1:]
        self.tickers = df.TICKER.unique()

        return len(self.times)-28-1, len(self.tickers)

    def get_indicators(self):
        self.conv3d = torch.nn.Conv3d(in_channels=4, out_channels=32,
        kernel_size=(1, 3, 1), padding="same").to('cuda')
        self.relu = torch.nn.ReLU()
        self.tucker = Tucker(rank=self.state_shape, init="random")
        self.m = 28
        self.w1, self.w2, self.w3 = 28, 14, 9
        
        df = (pd.DataFrame(self.stock_df, columns=self.tickers))
        df = df.dropna()
        mp = {ticker: pd.DataFrame(df[ticker]).rename(columns={ticker: "close"})
         for ticker in self.tickers}
        # SMA df
        sma = {ticker: pd.DataFrame(mp[ticker].ta.sma(self.w1)).rename(
          columns={"SMA_28": ticker}) for ticker in self.tickers}
        sma_df = pd.concat(sma.values(), axis=1).fillna(0)
        # RSI df
        rsi = {ticker: pd.DataFrame(mp[ticker].ta.rsi(self.w2)).rename(
          columns={"RSI_14": ticker}) for ticker in self.tickers}
        rsi_df = pd.concat(rsi.values(), axis=1).fillna(0)
        # MACD df
        macd = {ticker: pd.DataFrame(mp[ticker].ta.macd(self.w3, 26, 12)["MACD_9_26_12"])
        .rename(columns={"MACD_9_26_12": ticker}) for ticker in self.tickers}
        macd_df = pd.concat(macd.values(), axis=1).fillna(0)

        # Compute F = V @ Corr
        V = np.array([np.array(x.T) for x in [df, sma_df, rsi_df, macd_df]])
        Corr = np.array([np.corrcoef(x) for x in V])
        self.F = torch.from_numpy(np.einsum('aki,akj->akij', V, Corr)).to('cuda').float()


    def get_state(self) -> npt.NDArray[np.float64]:
        f = self.F[:, :, self.t + 28 - self.m : self.t + 28, :].clone().detach()      
        f = torch.unsqueeze(f, dim=0) 
        f = self.conv3d(f)
        f = self.relu(f)
        f = torch.squeeze(f, dim=0)
        f = f.cpu()
        core, _ = self.tucker.fit_transform(f)
        return core.detach().numpy()

    def get_prices(self) -> npt.NDArray[np.float64]:
        return np.append(self.stock_df.loc[self.times[self.t+28], :].to_numpy().
        flatten(), 1.0)
\end{minted}


