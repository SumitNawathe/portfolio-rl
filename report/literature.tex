\chapter{Literature Notes and Commentary}

\section{Differential Sharpe Ratio}

\cite{drl_mvo} utilizes the Differential Sharpe Ratio to implement and evaluate a reinforcement learning agent.
The Differential Sharpe Ratio is based on Portfolio Management Theory, and is developed in the author' previous works \cite{diff_sharpe_ratio_paper} and \cite{diff_sharpe_ratio_book}.
We briefly review the theory developed in both sources.

The traditional definition of the Sharpe Ratio is the ratio of expected excess returns to volatility.
If $R_t$ is the return of the portfolio at time $t$, and $r_f$ is the risk-free rate then
\begin{align*}
  \mathcal{S} = \frac{\mathbb{E}_t[R_t] - r_f}{\sqrt{\Var_t[R_t]}}
\end{align*}

This works well to analyze a strategy once all data is collected.
The goal of traditional portfolio theory is to maximize the Sharpe Ratio over the given time period (equivalently,
to maximize the mean-variance utility function).

Unfortunately, this will not work for a reinforcement learning agent. The agent must be given a reward after every time step,
but the traditional Sharpe ratio is only calculated at the end.

The Differential Sharpe Ratio attempts to remedy this by approximating a change in the total Sharpe ratio up to that point.
By summing together many of these incremental changes (though approximate), the cumulative rewards is an approximation of the total
Sharpe ratio over the complete time period.

The approximation works by updating moment-based estimators of the expectation and variance in the Sharpe Ratio formula.
Let $A_t$ and $B_t$ be estimates of the first and second moments of the return $R_t$ up to time $t$.
After time step $t$, having obtained $R_t$, we perform the following updates:
\begin{align*}
  \Delta A_t = R_t - A_{t-1} && A_t = A_{t-1} + \eta \Delta A_t \\
  \Delta B_t = R_t - B_{t-1} && B_t = B_{t-1} + \eta \Delta B_t
\end{align*}
where $A_0 = B_0 = 0$ and $\eta \sim 1/T$ is an update parameter, where there are $T$ total time periods.
These updates are essentially exponential moving averages.

Let $S_t$ be an approximation of the Sharpe Ratio up to time $t$ based on estimates $A$ and $B$. That is,
\begin{align*}
  S_t = \frac{A_t}{K_\eta \sqrt{B_t - A_t^2}} && K_\eta = \sqrt{\frac{1-\eta/2}{1-\eta}}
\end{align*}
The definition here ignores the risk-free rate term. $K_\eta$ is a normalization constant to ensure an unbiased estimator.

Pretend that at the update for time $t$, $A_{t-1}$ and $B_{t-1}$ are constants,
and $R_t$ is also a known constant. Then the updates to $A_t$ and $B_t$ really only depend on the time step / weight parameter $\eta$.
Indeed, if $\eta = 0$, then $A_t = A_{t-1}$ and $B_t = B_{t-1}$, so $S_t = S_{t-1}$.
Now consider varying $\eta$; expanding the Sharpe ratio estimator formula gives
\begin{align*}
  S_t \approx S_{t-1} + \eta \frac{\diff S_t}{\diff \eta}\Big|_{\eta = 0} + o(\eta^2)
\end{align*}

If $\eta$ is small, the final term is negligible, so this formula gives us an exponential-moving-average update for $S_t$.
The Differential Sharpe Ratio is defined to be proportional derivative in that expression. With some tedious calculus, we find that
\begin{align*}
  D_t = \frac{\diff S_t}{\diff \eta} = \frac{B_t \Delta A_t - \frac{1}{2} A_t \Delta B_t}{(B_t - A_t^2)^{3/2}}
\end{align*}

This reward function is simple to implement in an environment. The authors of the original papers provide
experimental support for the value of this reward function in a reinforcement learning setting.

